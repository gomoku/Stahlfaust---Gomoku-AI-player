\NeedsTeXFormat{LaTeX2e}
\documentclass[10pt]{article}

\addtolength{\hoffset}{-0.5cm}
\addtolength{\textwidth}{1cm}

\usepackage{a4wide,latexsym,amssymb,amsfonts,verbatim}
\usepackage[all]{xy}
\usepackage{pgf,pgfarrows,pgfnodes}

%\nonfrenchspacing


\usepackage[latin1]{inputenc}

\pagestyle{empty}

\newcommand{\N}{{\mathbb N}}
\newcommand{\Z}{{\mathbb Z}}
\newcommand{\R}{{\mathbb R}}
\newcommand{\Q}{{\mathbb Q}}
\newcommand{\C}{{\mathbb C}}

\renewcommand{\theequation}{\arabic{equation}}
%\renewcommand{\labelenumi}{(\alph{enumi})}

\newenvironment{solution}[1]{%
\vspace{0.5cm}
\begin{sloppypar}\noindent {\bf #1:}%
{\nopagebreak\hspace*{\fill}}
\end{sloppypar}\medskip}

\newcommand{\blattheader}[2]{{\bf \noindent
Marco Kunze ({\tt makunze@cs.tu-berlin.de})\hfill #2\\
Sebastian Nowozin ({\tt nowozin@cs.tu-berlin.de})\\[0.4cm]}

\begin{center}{{\Large \bf #1}}
\end{center}

\vspace*{0.3cm}
\noindent}

%\parindent=0cm

\usepackage{fancyhdr,lastpage}
\usepackage{a4wide,latexsym,amssymb,amsfonts,verbatim}
%\usepackage[dvips]{graphicx}
\usepackage{color}
\pagestyle{fancy}
\cfoot{ \thepage ~/ \pageref{LastPage} }
\renewcommand{\headrulewidth}{0pt}
\renewcommand{\footrulewidth}{0pt}



\begin{document}
\blattheader{An AI for Gomoku/Wuziqi\\
$\alpha-\beta$ and more...}{2004/12/29}
\begin{center}
How to think 20 moves ahead.
\end{center}

\section{Introduction}

This document details our solution to the second of two projects of the
``Artificial Intelligence'' course in fall 2004 at the SJTU.

Like the first project, the problem given to the students is easy to describe,
but difficult to solve.  The students are asked to write a program playing the
game of Wuziqi, which is also known by Gobang, free-stlye Gomoku or ``Five in
a row''.

\subsection{The game}

The standard playing rules are the following: On a board of 15x15 rows and
column two players, black and white, in turns place one stone of their color
on a free field until either the field is full or one player has won.  A
player has won if he has five or more stones of his color in a
straight row, be it horizontal, vertical or diagonal.\footnote{There are more
advanced rules in variations of this game, for example in ``standard Gomoku''
a player is only granted a win if he has {\em exactly} five stones in a row.
Also, professional ``Renju'' is a more complex variant restricting movements
even further.}

The game is interesting for two reasons.  The first is that the rules are
very easy to understand and everybody can play this game to some degree.  The
second reason - and the challenging part of the excercise - is that the game
can develop a deep strategy, with moves having to be planned ahead 10 or more
plys ahead to win successfully or even to just defend successfully against a
strong opponent.  World class Gomoku player plan ahead more than 25 plys.

Gomoku has a seemingly large search space due to many possible moves on a
15x15 board.  Additionally, it has been shown that generalized Gomoku is
PSPACE-complete, which means that Gomoku played on an board with $n$ fields,
the complexity increases without bounds for increasing $n$.  The net result
is, that without any clever tricks, Gomoku even on a 15x15 board is
computationally very difficult to solve.

Albeit this difficulties Gomoku has been solved by L. Victor Allis
in 1992~\cite{Allis92} by using a new method he coined ``dependency-based
search'', which we will describe in detail below.

We apply $\alpha-\beta$ search with some modifications and the
dependency-based search method to create a strong AI player, with no limit of
planning ahead for some situations of the game.


\section{Basic definitions}

We use a number of definitions in the following sections, which are explained
here.

\begin{itemize}
\item {\em Threat}.  A threat is a move to which you have to react.  The most
simple example is a {\em four}, which is a series of four same-coloured
stones, with one bordering field being free.  If the the opponent does not
react or does not win himself in the next move, the player owning the threat
has won.  We use a limited set of possible threats, which we will discuss in
detail now.  We assume the attacker always has the white color.  All cases are
shown with one example, but there are more cases than the one shown, because
of symmetry.

\begin{itemize}
\item {\em Five}.  A five is a simple row of five stones and the player has won.

\begin{center}
\begin{pgfpicture}{0cm}{0cm}{6cm}{2cm}
% (0cm,0cm) is the lower left corner,
% (5cm,2cm) is the upper right corner.
\pgfcircle[stroke]{\pgfxy(1,1)}{0.4cm}
\pgfcircle[stroke]{\pgfxy(2,1)}{0.4cm}
\pgfcircle[stroke]{\pgfxy(3,1)}{0.4cm}
\pgfcircle[stroke]{\pgfxy(4,1)}{0.4cm}
\pgfcircle[stroke]{\pgfxy(5,1)}{0.4cm}
\end{pgfpicture}
\end{center}

\item {\em Straight four}.  A straight four is a row of four stones with both
ends free.  If the other player has not won in his next move, this situation
is a sure win.

\begin{center}
\begin{pgfpicture}{0cm}{0cm}{6cm}{2cm}
\pgfcircle[stroke]{\pgfxy(1,1)}{0.05cm}
\pgfcircle[stroke]{\pgfxy(2,1)}{0.4cm}
\pgfcircle[stroke]{\pgfxy(3,1)}{0.4cm}
\pgfcircle[stroke]{\pgfxy(4,1)}{0.4cm}
\pgfcircle[stroke]{\pgfxy(5,1)}{0.4cm}
\pgfcircle[stroke]{\pgfxy(6,1)}{0.05cm}
\end{pgfpicture}
\end{center}

\item {\em Four}.  If the other player cannot win in his next move, he has to
defend against the four, which is a row of four stones, but with one end
closed by the opponent and the other end being free.

\begin{center}
\begin{pgfpicture}{0cm}{0cm}{6cm}{2cm}
\pgfcircle[fill]{\pgfxy(1,1)}{0.4cm}
\pgfcircle[stroke]{\pgfxy(2,1)}{0.4cm}
\pgfcircle[stroke]{\pgfxy(3,1)}{0.4cm}
\pgfcircle[stroke]{\pgfxy(4,1)}{0.4cm}
\pgfcircle[stroke]{\pgfxy(5,1)}{0.4cm}
\pgfcircle[stroke]{\pgfxy(6,1)}{0.05cm}
\end{pgfpicture}
\end{center}

\item {\em Three}.  The three is a more long term threat to the opponent and
although he has to react to it, he has more choice in the possible defending
moves to make.  If undefended, the three can be extended to a straight four,
which is a sure win.

\begin{center}
\begin{pgfpicture}{0cm}{0cm}{6cm}{2cm}
\pgfcircle[stroke]{\pgfxy(1,1)}{0.05cm}
\pgfcircle[stroke]{\pgfxy(2,1)}{0.4cm}
\pgfcircle[stroke]{\pgfxy(3,1)}{0.4cm}
\pgfcircle[stroke]{\pgfxy(4,1)}{0.4cm}
\pgfcircle[stroke]{\pgfxy(5,1)}{0.05cm}
\end{pgfpicture}
\end{center}

\item {\em Broken three}.  Using the broken three, the attacker can exert
great control about the opponents stones.  It has to be defended against, just
as with the three.

\begin{center}
\begin{pgfpicture}{0cm}{0cm}{6cm}{2cm}
\pgfcircle[stroke]{\pgfxy(1,1)}{0.05cm}
\pgfcircle[stroke]{\pgfxy(2,1)}{0.4cm}
\pgfcircle[stroke]{\pgfxy(3,1)}{0.05cm}
\pgfcircle[stroke]{\pgfxy(4,1)}{0.4cm}
\pgfcircle[stroke]{\pgfxy(5,1)}{0.4cm}
\pgfcircle[stroke]{\pgfxy(6,1)}{0.05cm}
\end{pgfpicture}
\end{center}
\end{itemize}

\item {\em Threat category}.  To each type of threat we associate a threat
category (or class), which equals the number of undefended moves necessary to
win.  The five has a class of 0, the straight four and the four have class 1
and the three and broken three have class 2.
\item {\em Threat sequence}.  A threat sequence is a series of single attacker
moves and one or more defending moves.  The last attacking move reaches a goal
state and has no defending moves.  A goal state is either a five or the 
undefendable straight four.
\end{itemize}


\section{Alpha-Beta search}

We use a standard four level (two pair move) alpha-beta search stage as
detailed in~\cite{Nilsson99}.  The state space searched is the board itself
and the operations are single moves.

\subsection{The static evaluation function}

The basic static evaluation function basically uses three measurements to rate a move:

\begin{enumerate}
\item The possibilities the move gives us to build rows of five stones in the future.

The space around the newly set stone is the lowest measurement we apply for
every move. That means, if no other criteria work, we will set the stone that
we, we can build most possible combinations of rows of five stones over it.

\item Building up long lines (or combinations of long lines) to work towards
threats and to prefer special kinds of threats.

This is done pretty easily: The row of own stones built up by the new stone is
counted and assigned by a multiplier. This for example has the advantage, that
a ``straight four'' will be preferred to a ``four'' with a hole.

\item The identification of a winning situation.

This is obvious.

\end{enumerate}

Besides the pure measurements and assigned points, the static evaluation
function has to be fast.

In order to need not to evaluate the whole board in every move, we just
consider the space around a newly set stone for the rating. This means, we look
at the 4 lines of 9 pieces which lead through the new stone, building up a
star. In order to avoid the calculation useless every time we want to rate a
move, we generate the values before the game starts into a table with something
around 260000 entries, which is much less and faster than doing this during the
execution of the alpha-beta algorithm.

This fast way of rating single nodes allows us not just to call the static
evaluation function in all the leaf nodes, but in every single node. During the
alpha-beta-search, the nodes are sorted (higher successors for MAX-nodes and
lower successors first for MIN-nodes) allowing a much faster and higher cut-off
rate during the alpha-beta-procedure~\cite{Nilsson99}.

\subsection{Supporting db-search: statically evaluating threats}

The alpha-beta method tries to support the db-search.  This is done by
favoring the creation of new high category threats.  Our assumption is that
other AI players will be unable to track a large number of high category
threats more successful than we are able to do using db-search.  Additionally
the large number of threats on the board will create many possible threat
sequences of which we hope there will be a winning one.

During the static evaluation function, threats therefore are higher rated than
anything else (except win of course), and in threats of the same category,
longer lines of own stones are preferred.

Since the threat detection for the static evaluation function has to be much
faster, but less generalized than during the db-search, it has it's own methods
for this purpose, allowing a fast pattern matching on the same local lines as
for the normal static evaluation function, also using a cached look-up table.

\subsection{Keeping the branching factor small: finding interesting fields}

For the speed of the alpha-beta-procedure, it is crucial to keep the branching
factor small. Since the branching factor is decided by the new number of
interesting moves (which means the number of moves we should consider to take
in the next step) we have to find ways to exclude as many fields as possible.

This is done in three steps:

\begin{enumerate}
\item Using ``binary dilation'' on the board.

Binary dilation, originally coming from the field of digital image processing,
allows us to select just the fields around already set stones, using a fixed
pattern. This way we just consider the 16 fields (starformed) around already
set stones as interesting.

\item Looking for threats.

If we are already forced to block threats, we don't need to consider other
fields than the blocking ones, except...

\item Looking for own possible threats.

If we are able to build up threats of lower categories than the opponent
threats, we can force him before he turns his threat into a win, so these moves
can be considerer.
\end{enumerate}

This method allows us crucial branching limitations in threat situation (which
we are looking for) and in end-game situations. The threats and interesting
moves ar also locally evaluated, never have to been recalculated for the whole
board, to let us be fast.

\section{Dependency-based search}

Dependency-based search is a single-agent search algorithm to explore a space
state under certain constraints.  It was invented by Victor Allis specifically
to solve thinking games such as Connect-four and Gomoku.

Now, following~\cite{Allis92} we will discuss two points that are crucial for
the understanding of how we use db-search in Gomoku.  The first is how
the adversary-agent problem of Gomoku is converted to a single-agent search
space.  The second is how db-search works on this single-agent search space.

\subsection{Converting Gomoku to a single-agent search space}

The idea of converting the adversary search space to a single agent one, is a
simple and clever one: we allow the defender to make all possible defending
moves at once.  Then, every threatening move is combined with all the possible
moves to defend against this threat and the whole pair-move is combined as one
operation in the search space.

Of course, allowing the opponent to make more than one move at a time will
give him a huge advantage.  But the advantage to safely plan ahead a large
number of moves far outweights this disadvantage.  That is, we are denied the
ability to recognize a large number of cases where we could have won, but
without this simplification we would not have the computational ability
anyway.  What remains is a small portion of the possible cases from which we
can try to find a win.

\subsection{Applying db-search}

The theoretic framework for db-search provided in~\cite{Allis92} is quite
large and generic.  We implemented this framework completely and besides
solving the Gomoku with it, we solved a simple problem known as ``Double
Letter Puzzle'' for testing purposes.  Below we will shortly describe the
operation of db-search, although a larger discussion would take to much time.

\subsubsection{The basic idea}

The basic idea of db-search is to use operators to transfer between states in
the state space.  The already discovered states are kept in a directed acyclic
graph (DAG).  The constraints to apply an operator on a state already
discovered are the following:
\begin{itemize}
\item The operator is valid.

The operators preconditions are valid on the given state and the operator
can be applied.
\item The application of the operator {\em depends} on the previous one.

This condition is always assumed true if the state is the root state, as there
was no previous operator.

Otherwise, if there was a previous operator leading to this state, the
application of the new operator examined has to depend on the result of the
previous operator.  That is, without the previous operator applied before, the
operator would not work.  This is one important part where useless
explorations of the space state are limited.
\item The application of the operator has to {\em create new dependent
operators}.

The exploration of the state space is greatly limited by this condition.  Only
operators are applied, if they depend on the previous operator {\em and} if
their application creates at least one operator that is dependent of the
result of the operator considered.
\end{itemize}

\subsection{Implementation}

The db-search description is quite simple and intuitive, but an efficient
implementation of this semantics is another matter.  One pseudo-code
implementation is given in the original paper by Allis, which we used as basic
design guideline for our from-scratch implementation.  The really difficult
part in implementing db-search were a number of predicates (such as
\verb|NotInConflict| given in the original paper) and meta-operators
(such as \verb|Combine|).  The implementation we use works in two stages, a
{\em dependency following} and a {\em state combination} stage.  An
in-depth discussion would take too much time, but suffice to say, the three
conditions listed above - also known as ``meta operator'' used in the
theoretical db-search framework - are ensured by successively executing this
stages in pairs, called levels.

For Gomoku, we often have seen one single application of the dependency
following stage following more than 10 moves in a row.  The largest sequences
we have seen being checked extended more than 25 moves.  In no case observed
by us the search has required more than four levels.

On an algorithmic level our implementation is very space and time efficient,
often exploring less than a thousand states and taking less than a few seconds
to explore all possible threat trees.  However, we did not profile and
manually optimize the code so far, which could still result in improvements of
an order of magnitude.

Note that the winning threat sequence search is not complete.  This means it
will not find all winning threat sequences or miss them while searching.
However, if it finds one, it guarantees it is not refutable.  As such, it is
used only in conjunction with the $\alpha-\beta$ search, not as replacement.

One caveat: we did not implement complete three- and four-state combinations
in the db-search combination stage, which can lead to problems, see below.
This is because we ran out of time to devise an efficient implementation and
our implementation already took considerable time to complete.


\subsection{Winning threat sequence search}

The winning threat sequence search is called with a board position.  The goal
condition for the search to succeed is a {\em five} or {\em straight four}
pattern on the board.  As long as this pattern is not found and the search
space is not exhausted, we continue to search.

Also, as an extra condition to ensure timely execution, we added a timeout
mechanism.  If the timeout given in milliseconds is exceeded, the search is
cancelled and no winning threat sequence is returned.

The basic operation of the search is the following:

\begin{enumerate}
\item Search for potential winning threat sequences

The db-search is started with the board as parameter and proceeds expanding
the search tree using the given constraints.  For each new state reached it is
checked if it is a goal state.  If it is no goal state, the search proceeds.
If it is a goal state, a potential winning threat sequence is extracted by
converting all the operators on the path from the root state to the goal state
into moves.  Then

\item For each potential winning threat sequence found, we check if it can be
defended against by the opponent.  Defending means, that by making a good
choice on his defending moves, he can build threat sequences himself.  His
threat sequences are a danger to us for two reasons:
\begin{itemize}
	\item They could lead to a win for the opponent.
	\item They could lead to occupation of squares we would need ourselves
later in our threat sequence or occupy defending squares ahead of time.
\end{itemize}

To check for this, we start a second db-search, this time trying to defend
against our own attacks.  Note, while the attack sequence search we did for
ourselves is not complete, this defending search has to be complete to be sure
our attack is working every time.  See the next subsection for an explanation
why this is so.

The defending db-search works like this:
\begin{enumerate}
	\item Make an attacker move
	\item If the attacker has won with this move, we mark the sequence as
unrefutable.
	\item Try to find a potential winning threat sequence, but limit the
applicable threats to a class one lower than the attacker threat.

It would make no sense to search for threats with a class as high as or higher
than the attacker has challenged us with, as a challenge with a class as high
as his would mean he is always a class lower than we are in the next move,
leading to a win for the attacker.  On the other hand, for example he
challenges us with a class 2 threat, we can successfully defend ourselves with
both a class 1 (if he does not react, we win in the next move) and class 0 (we
won) threat.
	\item If the threat sequences are potential winning sequences or if they
lead to an occupation of any attacker or defender square later in the
attackers threat sequence, we return and mark the threat sequence as refuted.
	\item If there are still moves left in the attackers threat sequence,
advance to the next move and start at (a).
\end{enumerate}

If a winning threat sequence has been found, we abort the search and return
early with the sequence.
\end{enumerate}

\subsubsection{Completeness of the defending db-search}

The defending threat sequence search has to be complete in that it never
misses any defending sequence the opponent can make.  This requires
completeness in the db-search algorithm.  However, as we have seen that the
attacker winning threat sequence search is not complete, how can we make the
defending one complete?

This is achieved by restricting the number of applicable operators
drastically: only operators with a category below the attackers one are
allowed.  As the highest category is 2, for threats that will win in two moves
if not defended against, the maximum possible threat category the defender can
apply is 1.  However, to all threats of class 1, there is always a reply using
only one stone, which does not require our ``multiple-defender-stone''
assumption, which makes the attackers search incomplete.  Hence the defending
search is complete.

However, in our implementation a small part of the db-search algorithm is
missing.  This leads to some rare cases not being explored in the search tree
although they should be.  We have seen no such incident in a real game, but
there are artificial situations in which this happens.  As this is a
vulnerability in our AI player, we will not describe it in too much detail
here which situations could trigger this, but we think they are unlikely to
occur in a real game our player plays.

\subsection{In game operation}

The winning threat sequence search is called whenever an opponent move has
been made.  Then, after the alpha-beta and interesting moves module is ran, we
use the remaining time to run a time-constrained db-search, giving the current
board position.

If the winning threat sequence finds no winning threat sequence, the
alpha-beta result is used.  If it does find a sequence, this guarantees us a
win:

\begin{itemize}
\item We have won in no more than $n$ moves, where $n$ is the depth given in
the threat sequence plus the number of opponent threats on the board.
\item We have a move sequence which the opponent cannot refute at the last
move.
\item We know all possible defender moves in advance that can delay our win.
If the defender does not make any of this moves, we fall back to
$\alpha-\beta$ search and finding a new winning threat sequence.  In almost
all cases the $\alpha-\beta$ search will find a guaranteed win already, as the
latest threat was not defended against.  In rare cases it might not find a
good move and a new much shorter threat sequence is returned by the winning
threat sequence search.
\end{itemize}

The threat sequence is stored and followed as long as the defender makes one
of the known defending moves.  We expect almost all opponent AIs to blindly
following the threat sequences expected defending moves until at least near
the end.

\begin{thebibliography}{99}
\bibitem{Allis92} L. Victor Allis, ``Searching for Solutions in Games and
Artificial Intelligence'', Ph.D. thesis available online at
\verb|http://www.cs.vu.nl/~victor/thesis.html| and available as book (ISBN
90-9007488-0)
\bibitem{Nilsson99} Nils J. Nilsson, ``Artificial Intelligence - A New
Synthesis'', book (ISBN 7-111-07438-6)
\end{thebibliography}
\end{document}
